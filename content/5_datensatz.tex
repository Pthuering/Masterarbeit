\section{Datensatzerstellung und -aufbereitung} \label{chap:5}

% Ziel: 8-10 Seiten

    \subsection{Datenerhebung}
    \label{sec:5.1}
    
    % Ziel: 2-3 Seiten
    % - Sammlung bestehender Verkehrsanweisungen bei LVB
    % - Charakterisierung des Rohdatenbestands


    \subsection{Datensatzentwicklung für Fine-Tuning}
    \label{sec:5.2}
    
    % Ziel: 6-7 Seiten
    % KERNKAPITEL - hier entsteht die Basis für das Fine-Tuning
    
        \subsubsection{Annotationsstrategie}
        \label{sec:5.2.1}
        
        % - Input-Output-Paare erstellen
        % - Verschiedene Zielformate
        % - Qualitätssicherung der Annotationen
        
        
        \subsubsection{Datenaugmentierung}
        \label{sec:5.2.2}
        
        % - Formatvariationen
        % - Synthetische Beispiele
        
        
        \subsubsection{Datensatzstruktur}
        \label{sec:5.2.3}
        
        % - Train-Validation-Test-Split
        % - Größe und Verteilung
        
        
        \subsubsection{Herausforderungen bei kleinem Datensatz}
        \label{sec:5.2.4}
        
        % - Strategien aus der Literatur (multimodale Integration, stufenweise Optimierung)
        % - Übertragung auf diese Arbeit
        % - Training verschiedener Parameterebenen

