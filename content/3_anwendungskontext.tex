\section{Anwendungskontext und Anforderungen} \label{chap:3}

% Ziel: 8-10 Seiten

    \subsection{Leipziger Verkehrsbetriebe}
    \label{sec:3.1}
    
    % Ziel: 2-3 Seiten
    % - Relevante Organisationsstrukturen
    % - Aktueller Workflow: Manuelle Transformation von Verkehrsanweisungen
    % - Zielgruppen und Kommunikationskanäle
    % - Herausforderungen und Ineffizienzen


    \subsection{Analyse der Verkehrsanweisungen}
    \label{sec:3.2}
    
    % Ziel: 4-5 Seiten
    % - Detaillierte Charakterisierung des Problems
    % - Struktur technischer Verkehrsanweisungen
    % - Beispiele und Komplexität
    % - Zielformate für verschiedene Nutzergruppen
    % - Linguistische Besonderheiten
    
        \subsubsection{Kategorisierung von Verkehrsmeldungen nach Komplexität}
        \label{sec:3.2.1}
        
        % KOMPLEXITÄTSKATEGORIEN VON VERKEHRSMELDUNGEN
        % Die Verkehrsanweisungen der LVB lassen sich nach Art und Umfang der betroffenen
        % Linien und Maßnahmen in verschiedene Kategorien einteilen:
        %
        % KATEGORIE 1: EINZELLINIE, EINE RICHTUNG
        % - Eine Linie in eine Richtung betroffen
        % - Einfachste Struktur
        % - Beispiel: "Linie 3 fährt Richtung Taucha über Umleitung"
        %
        % KATEGORIE 2: EINZELLINIE, MEHRERE RICHTUNGEN MIT UNTERSCHIEDLICHEN MASSNAHMEN
        % - Eine Linie in mehrere Richtungen betroffen
        % - Jede Richtung hat unterschiedliche Umleitungen/Maßnahmen
        % - Darstellung: Richtungsspezifische Maßnahmen werden hintereinander formuliert
        % - Reihenfolge: Erst Richtung A, dann Richtung B
        % - Beispiel: "Linie 11 fährt Richtung Schkeuditz über X, Richtung Markkleeberg über Y"
        %
        % KATEGORIE 3: EINZELLINIE, BEIDE RICHTUNGEN MIT GLEICHEN MASSNAHMEN
        % - Eine Linie in beide Richtungen betroffen
        % - Identische Maßnahmen für beide Richtungen
        % - Beispiel: "Linie 7 wird in beiden Richtungen umgeleitet"
        %
        % KATEGORIE 4: MEHRERE LINIEN MIT GLEICHEN MASSNAHMEN
        % - Mehrere Linien von identischen Maßnahmen betroffen
        % - Darstellung: Aufzählung der betroffenen Linien
        % - Beispiel: "Die Linien 1, 2 und 3 werden über Straße X umgeleitet"
        %
        % KATEGORIE 5: MEHRERE LINIEN MIT UNTERSCHIEDLICHEN MASSNAHMEN
        % - Mehrere Linien, jede mit individuellen Maßnahmen
        % - Hohe Komplexität durch mehrfache Differenzierung
        % - Beispiel: "Linie 3 über Route A, Linie 7 über Route B, Linie 11 verkehrt nicht"
        %
        % KATEGORIE 6: MEHRERE ZEITRÄUME/BAUPHASEN
        % - Maßnahmen ändern sich über verschiedene Zeiträume
        % - Unterschiedliche Regelungen pro Bauphase
        % - Beispiel: "Phase 1 (01.-15.03.): Linie 3 umgeleitet; Phase 2 (16.-31.03.): zusätzlich Linie 7"
        %
        % KATEGORIE 7: KOMBINATION AUS ZEITRÄUMEN UND MEHREREN LINIEN
        % - Höchste Komplexität
        % - Verschiedene Linien mit unterschiedlichen Maßnahmen über mehrere Zeiträume
        %
        % DARSTELLUNGSKONVENTIONEN:
        % - Maßnahmen im gleichen Zeitraum werden in einem Text hintereinander zusammengefasst
        % - Mehrere Linien mit gleicher Maßnahme werden durch Aufzählung zusammengefasst
        % - Richtungsspezifische Maßnahmen werden hintereinander formuliert (erst Richtung A, dann B)
        % - Verschiedene Zeiträume/Bauphasen werden explizit gekennzeichnet
        %
        % BEISPIEL KOMBINATION (Kategorie 7):
        % "Vom 1. bis 15. März fahren die Linien 1 und 2 Richtung Nord über Umleitung X,
        %  Richtung Süd über Umleitung Y. Linie 3 verkehrt nicht. Ab 16. März fahren alle
        %  Linien wieder regulär."
        %
        % RELEVANZ FÜR MODELLTRAINING:
        % - Diese Kategorien bestimmen die Diversität des Trainingsdatensatzes (siehe Abschnitt~\ref{sec:5.2.3})
        % - Unterschiedliche Komplexitätsstufen erfordern differenzierte Transformationsstrategien
        % - Verteilung der Kategorien beeinflusst Modellperformance


    \subsection{Anforderungen an das System}
    \label{sec:3.3}
    
    % Ziel: 2 Seiten
    
        \subsubsection{Funktionale Anforderungen}
        \label{sec:3.3.1}
        
        % - Automatische Transformation mit minimaler Nutzerinteraktion
        % - Konsistente Formulierungen
        % - Mehrere Ausgabeformate
        % - Qualitätssicherung
        
        
        \subsubsection{Nicht-funktionale Anforderungen}
        \label{sec:3.3.2}
        
        % - Datenschutz (DSGVO-Konformität)
        % - Ressourceneffizienz
        % - Integrationsfähigkeit in bestehende Systeme
        % - Wartbarkeit und Skalierbarkeit

