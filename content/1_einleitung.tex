\section{Einleitung} \label{chap:1}

% Ziel: 8-9 Seiten

    \subsection{Motivation und Problemstellung}
    \label{sec:1.1}
    
    % Ziel: 2-3 Seiten
    % Inhalt:
    % - Manuelle Transformation bei LVB zeitaufwändig und inkonsistent
    % - Potenzial durch KI-Automatisierung
    % - Problematik überdimensionierter KI-Systeme in der Praxis


    \subsection{Zielsetzung und Forschungsfragen}
    \label{sec:1.2}
    
    % Ziel: 2 Seiten
    % Hauptziel: Automatische Transformation technischer Verkehrsanweisungen in nutzergerechte 
    % Formate durch Fine-Tuning eines kompakten Sprachmodells
    %
    % Forschungsfragen:
    % 1. Wie kann ein 7B-Parameter-Modell durch Fine-Tuning für diese spezifische Aufgabe optimiert werden?
    % 2. Welche Qualität erreichen quantisierte Modellversionen im Vergleich zum Vollmodell?
    % 3. Wie kann die Transformation durch Prompt Engineering automatisiert werden?


    \subsection{Wissenschaftlicher Beitrag und Forschungslücke}
    \label{sec:1.3}
    
    % Ziel: 2-3 Seiten
    % Einordnung der Arbeit:
    % - Domänenspezifisches Fine-Tuning für Verkehrsinformationen
    % - Praxisnahe Evaluation quantisierter Modelle
    % - Open-Source-Ansatz mit deutschem Modell (LeoLM)
    %
    % Forschungslücke:
    % - Keine Untersuchungen zu ressourceneffizienten Modellen für Verkehrsanweisungs-Transformation
    % - Fehlende Evaluation quantisierter Modelle in dieser Domäne
    % - Deutschsprachige Modelle im Verkehrssektor unterrepräsentiert
    % - Vergleich ressourceneffizienter Ansätze vs. generische Großmodelle


    \subsection{Aufbau der Arbeit}
    \label{sec:1.4}
    
    % Ziel: 1 Seite
    % Kurze Übersicht über die Struktur der Arbeit

